\documentclass[a4paper]{article}

\usepackage{hyperref}

\title{Using local binary patterns to read license plates in photographs}

% Paragraph indentation
\setlength{\parindent}{0pt}
\setlength{\parskip}{1ex plus 0.5ex minus 0.2ex}

\begin{document}
\maketitle

\section*{Project members}
Gijs van der Voort\\
Richard Torenvliet\\
Jayke Meijer\\
Tadde\"us Kroes\\
Fabi\'en Tesselaar

\tableofcontents
\setcounter{secnumdepth}{1}

\section{Problem description}

License plates are used for uniquely identifying motorized vehicles and are
made to be read by humans from great distances and in all kinds of weather
conditions.

Reading license plates with a computer is much more difficult. Our dataset
contains photographs of license plates from various angles and distances. This
means that not only do we have to implement a method to read the actual
characters, but also have to determine the location of the license plate and
its transformation due to different angles.

We will focus our research on reading the transformed characters on the
license plate, of which we know where the letters are located. This is because
Microsoft recently published a new and effective method to find the location of
text in an image.

Determining what character we are looking at will be done by using Local Binary
Patterns. The main goal of our research is finding out how effective LBPs are
in classifying characters on a licenseplate.

In short our program must be able to do the following:

\begin{enumerate}
    \item Use perspective transformation to obtain an upfront view of license
          plate.
    \item Reduce noise where possible.
    \item Extract each character using the location points in the info file.
    \item Transform character to a normal form.
    \item Create a local binary pattern histogram vector.
    \item Match the found vector with a learning set.
\end{enumerate}

\section{Solutions}

Now that the problem is defined, the next step is stating our basic solutions.
This will come in a few steps as well.

\subsection{Transformation}

A simple perspective transformation will be sufficient to transform and resize
the plate to a normalized format. The corner positions of license plates in the
dataset are supplied together with the dataset.

\subsection{Reducing noise}

Small amounts of noise will probably be suppressed by usage of a Gaussian
filter. A real problem occurs in very dirty license plates, where branches and
dirt over a letter could radically change the local binary pattern. A question
we can ask ourselves here, is whether we want to concentrate ourselves on these
exceptional cases. By law, license plates have to be readable. Therefore, we
will first direct our attention at getting a higher score in the 'regular' test
set before addressing these cases. Considered the fact that the LBP algorithm
divides a letter into a lot of cells, there is a good change that a great
number of cells will still match the learning set, and thus still return the
correct character as a best match. Therefore, we expect the algorithm to be
very robust when dealing with noisy images.

\subsection{Extracting a letter}

Because we are already given the locations of the characters, we only need to
transform those locations using the same perspective transformation used to
create a front facing license plate. The next step is to transform the
characters to a normalized manner. The size of the letter W is used as a
standard to normalize the width of all the characters, because W is the widest
character of the alphabet. We plan to also normalize the height of characters,
the best manner for this is still to be determined.

\begin{enumerate}
    \item Crop the image in such a way that the character precisely fits the
          image.
    \item Scale the image to a standard height.
    \item Extend the image on either the left or right side to a certain width.
\end{enumerate}

The resulting image will always have the same size, the character contained
will always be of the same height, and the character will alway be positioned
at either the left of right side of the image.

\subsection{Local binary patterns}

Once we have separate digits and characters, we intent to use Local Binary
Patterns to determine what character or digit we are dealing with. Local Binary
Patters are a way to classify a texture based on the distribution of edge
directions in the image. Since letters on a license plate consist mainly of
straight lines and simple curves, LBP should be suited to identify these.

To our knowledge, LBP has yet not been used in this manner before. Therefore,
it will be the first thing to implement, to see if it lives up to the
expectations. When the proof of concept is there, it can be used in the final
program.

Important to note is that due to the normalization of characters before
applying LBP. Therefore, no further normalization is needed on the histograms.

Given the LBP of a character, a Support Vector Machine can be used to classify
the character to a character in a learning set. The SVM uses

\subsection{Matching the database}

Given the LBP of a character, a Support Vector Machine can be used to classify
the character to a character in a learning set. The SVM uses the collection of
histograms of an image as a feature vector.  The SVM can be trained with a
subsection of the given dataset called the ''Learning set''. Once trained, the
entire classifier can be saved as a Pickle object\footnote{See
\url{http://docs.python.org/library/pickle.html}} for later usage.

\section{Implementation}

In this section we will describe our implementations in more detail, explaining
choices we made.

\subsection{Licenseplate retrieval}

In order to retrieve the license plate from the entire image, we need to
perform a perspective transformation. However, to do this, we need to know the 
coordinates of the four corners of the licenseplate. For our dataset, this is
stored in XML files. So, the first step is to read these XML files.\\
\\
\paragraph*{XML reader}



\paragraph*{Perspective transformation}
Once we retrieved the cornerpoints of the licenseplate, we feed those to a
module that extracts the (warped) licenseplate from the original image, and
creates a new image where the licenseplate is cut out, and is transformed to a
rectangle.

\subsection{Noise reduction}

The image contains a lot of noise, both from camera errors due to dark noise 
etc., as from dirt on the license plate. In this case, noise therefore means 
any unwanted difference in color from the surrounding pixels.

\paragraph*{Camera noise and small amounts of dirt}
The dirt on the licenseplate can be of different sizes. We can reduce the 
smaller amounts of dirt in the same way as we reduce normal noise, by applying
a gaussian blur to the image. This is the next step in our program.\\
\\
The gaussian filter we use comes from the \texttt{scipy.ndimage} module. We use
this function instead of our own function, because the standard functions are
most likely more optimized then our own implementation, and speed is an
important factor in this application.

\paragraph*{Larger amounts of dirt}
Larger amounts of dirt are not going to be resolved by using a Gaussian filter.
We rely on one of the characteristics of the Local Binary Pattern, only looking
at the difference between two pixels, to take care of these problems.\\
Because there will probably always be a difference between the characters and
the dirt, and the fact that the characters are very black, the shape of the
characters will still be conserved in the LBP, even if there is dirt
surrounding the character.

\subsection{Character retrieval}

The retrieval of the character is done the same as the retrieval of the license
plate, by using a perspective transformation. The location of the characters on
the licenseplate is also available in de XML file, so this is parsed from that
as well.

\subsection{Creating Local Binary Patterns and feature vector}



\subsection{Classification}



\section{Finding parameters}

Now that we have a functioning system, we need to tune it to work properly for
license plates. This means we need to find the parameters. Throughout the 
program we have a number of parameters for which no standard choice is
available. These parameters are:\\
\\
\begin{tabular}{l|l}
	Parameter 			& Description\\
	\hline
	$\sigma$  			& The size of the gaussian blur.\\
	\emph{cell size}	& The size of a cell for which a histogram of LBPs will
	                      be generated.\\
	$\gamma$			& Parameter for the Radial kernel used in the SVM.\\
	$c$					& The soft margin of the SVM. Allows how much training
						  errors are excepted.
\end{tabular}\\
\\
For each of these parameters, we will describe how we searched for a good
value, and what value we decided on.

\subsection{Parameter $\sigma$}

The first parameter to decide on, is the $\sigma$ used in the Gaussian blur. To
find this parameter, we tested a few values, by checking visually what value
removed most noise out of the image, while keeping the edges sharp enough to
work with. By checking in the neighbourhood of the value that performed best,
we where able to 'zoom in' on what we thought was the best value. It turned out
that this was $\sigma = ?$.

\subsection{Parameter \emph{cell size}}

The cell size of the Local Binary Patterns determines over what region a
histogram is made. The trade-off here is that a bigger cell size makes the
classification less affected by relative movement of a character compared to
those in the learningset, since the important structure will be more likely to
remain in the same cell. However, if the cell size is too big, there will not
be enough cells to properly describe the different areas of the character, and
the featurevectors will not have enough elements.\\
\\
In order to find this parameter, we used a trial-and-error technique on a few
basic cell sizes, being ?, 16, ?. We found that the best result was reached by
using ??.

\subsection{Parameters $\gamma$ \& $c$}

The parameters $\gamma$ and $c$ are used for the SVM. $c$ is a standard
parameter for each type of SVM, called the 'soft margin'. This indicates how
exact each element in the learning set should be taken. A large soft margin
means that an element in the learning set that accidentally has a completely
different feature vector than expected, due to noise for example, is not taken
into account. If the soft margin is very small, then almost all vectors will be
taken into account, unless they differ extreme amounts.\\
$\gamma$ is a variable that determines the size of the radial kernel, and as
such blablabla.\\
\\
Since these parameters both influence the SVM, we need to find the best
combination of values. To do this, we perform a so-called grid-search. A
grid-search takes exponentially growing sequences for each parameter, and
checks for each combination of values what the score is. The combination with
the highest score is then used as our parameters, and the entire SVM will be
trained using those parameters.\\
\\
We found that the best values for these parameters are $c=?$ and $\gamma =?$.

\section{Results}

The wanted to find out two things with this research: The speed of the
classification and the accuracy. In this section we will show our findings.

\subsection{Speed}

Recognizing license plates is something that has to be done with good speed,
since there can be a lot of cars passing a camera, especially on a highway.
Therefore, we measured how well our program performed in terms of speed. We
measure the time used to classify a license plate, not the trainign of the
dataset, since that can be done offline, and speed is not a primary necessity
there.\\
\\
The speed of a classification turned out to be blablabla.

\subsection{Accuracy}

Of course, it is vital that the recognition of a license plate is correct,
almost correct is not good enough here. Therefore, we have to get the highest
accuracy score we possibly can. According to Wikipedia
\footnote{
\url{http://en.wikipedia.org/wiki/Automatic_number_plate_recognition}},
commercial license plate recognition software score about $90\%$ to $94\%$,
under optimal conditions and with modern equipment.\\
\\
Our program scores an average of blablabla.

\section{Conclusion}



\end{document}
